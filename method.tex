% !TEX root = CI_Adoption.tex

\section{Methods}
\label{sec:method}

\subsection{Data Gathering}

To perform our study, first and foremost, we should make sure our studied projects have sufficient history (e.g. 9*30 days) both before using Travis-CI and after using Travis-CI.  As the data were extracted from the inception until 2016-05-01, the projects selected should be created at least before 2014-11-08, so that there are at least 9*30*2 days from its creation to 2016-05-01. 
From GitHub Search API, we first found over 300k Java projects created before 2014-11-08 and with $\geqslant 500$ kilobytes. After consulting Travis-CI, we further identify projects that: 1) adopted Travis-CI; 2) at least 9 * 30 days from project creation to Travis-CI adoption; 3) at least 9 * 30 days from Travis-CI adoption to 2016-05-01. This filtering process left us 1566 projects.  
For each project, we collect the whole history of commits, issues, and pull requests from GitHub API, and the history of Travis-CI builds and job logs from Travis-CI repository using the ruby library of \textit{travis gem}.  In addition to these publicly available data, we further gathered the information about the number of tests and types of errors during the Travis-CI run.

% !TEX root = ../CI_Adoption.tex

\begin{table}[t] \centering
\small
  \caption{Examples for the outputs in terms of MAVEN, GRADLE and ANT
  \vspace{-0.2cm}
  }
  \label{log_example}

\begin{tabular}{ p{8cm}}
	\hline
	\\[-1.8ex]\hline
		\textbf{MAVEN}         \\
		\begin{tabular}[c]{@{}l@{}}
			\textit{-------------------------------------------------------}\\ 
			\textit{ T E S T S}\\ \textit{-------------------------------------------------------}\\ 
			\textit{Running org.sonar.ide.intellij.inspection.InspectionUtilsTest}\\ \textit{Tests run: 1, Failures: 0, Errors: 0, Skipped: 0, Time elapsed:}\\ \textit{0.202 sec}\\ 
			\textit{Results:}\\ 
			\textit{Tests run: 1, Failures: 0,Errors: 0, Skipped: 0}\end{tabular} \\  \hline
	
		\textbf{ANT}          \\
		\begin{tabular}[c]{@{}l@{}}
			\textit{{[}junit{]} Running limelight.BufferedImagePoolTest}\\ 
			\textit{{[}junit{]} Testsuite: limelight.BufferedImagePoolTest}\\ \textit{{[}junit{]} Tests run: 5, Failures: 0, Errors:0, Time elapsed: 0.143 sec}\end{tabular}                                                                                                                                                \\  \hline
		\textbf{GRADLE}          \\
		\begin{tabular}[c]{@{}l@{}}
			\textit{:test}\\ 
			\textit{...}\\ 
			\textit{1 test completed, 1 failed}\\
			\textit{:test FAILED}\\ 
			\textit{Total time: 3.8 secs}\end{tabular}      \\ \hline                                                                                                                                                                                                                                                
	\end{tabular}
\end{table}
\emph{Number of tests}: 
Each build has at least one job which indicates an execution of the build on a specified environment. Once the job is started, a tracking log will be generated, which records the detailed information of the build lifecycle including the installation steps and the build scripts for testing. To investigate the testing evolution across builds, we attempt to mine data from these logs. We build an automatic tool to analyze Travis-CI build logs and extract summary information about test executions. Table~\ref{log_example} shows the output examples for the three widely used Java build tools, \ie Maven, Ant and Gradle. We recognized how many tests have been executed from the outputs. Since the relationship between builds and jobs are one-to-many, we use the maximum number of tests as the test count for the build. 

% !TEX root = CI_Adoption.tex

\begin{table}[t] \centering
\small
  \caption{Travis-CI build failure taxonomy
  \vspace{-0.2cm}
  }
  \label{error_types}

\begin{tabular}{ p{1.5cm}  p{2.5cm}  p{4cm} }
	
\hline 
\\[-1.8ex]\hline
Category & Description & Examples of textual patterns \\ \hline 
\emph{failed test} & tests were failed & tests failed, TestsFailedException, tests unsuccessful\\ \hline
\emph{skipped or pending test} & tests were skipped or set to be pending & skipped tests \\ \hline
\emph{missing file or dependency} & required files or dependencies are not available &  FileNotFoundException, no such file to load, is not installed \\ \hline
\emph{code quality} & the code failed to satisfy the code standards & line too long, missing whitespace around operator, too many pylint violations \\ \hline
\emph{compile error} &compilation errors happened & Compilation failed, syntax check failed, attributeError, parse error\\ \hline
\emph{execution error} &an error occurs during the execution of code & runtime error, execution failed, test errors, OutOfMemoryError\\ \hline
\emph{time out} & the build could not be completed within the required time&“test run exceeded * minutes”, “..took longer than”, command time-out \\ \hline 
\emph{other} &other infrequent errors are included in this type & warnings on documentation, Aborted due to warnings, The remote end hung up unexpectedly \\
\hline

\end{tabular}

\end{table}


\emph{Error classification for failed builds}: 
In Travis-CI, a build will be marked as \textit{failed} if one of its \textit{non-allow-failure} jobs failed. From this point, to understand why the Travis-CI builds failed, we attempt to find out what happened in the jobs. We first manually reviewed a set of failed jobs to recognize what kinds of failures occur and the corresponding used textual patterns, and then extracted a group of mapping rules to classify these patterns into categories. To mitigate the risk of bias arising from missing and incorrect classification, we augmented the reviewing rates to improve our classification rules and remove spurious classification as much as possible. The classification scheme evolved over the process of manual review, and was refined gradually to cover more textual patterns. Finally, we summarize a group of mapping rules to category the errors into 8 types, as shown in Table~\ref{error_types}. 
We implemented tools to do the classification automatically. The detailed process is that: 1) failure log location. We first recognized failed jobs with \textit{non-allow-failure} attribute, which result in the build's final \textit{failed} state. Then we located the command which may indeed caused the breakdown. From the log file, we found the failed command which exited with a non-zero value.The log of this command usually described the detailed information about the deadly errors happened. If such commands are not recorded in the log file, we will expand search scope to the \textit{script} phase and the phase \textit{after script} based on Travis-CI build lifecycle. Because if errors occur before script phase, the job will be marked as \textit{errored} and stops immediately. Only if the \textit{script} phase returns a non-zero exit code or the phase \textit{after script} times out, the job will be marked as \textit{failed}. So we first check if there is a time-out error in \textit{after script}. If not, the errors happened in the \textit{script} phase potentially caused the build broken. 
2) log parsing. After extracting the log fragment which described the errors occurred, we use textual pattern matching method to recognize the errors. 
3) classification. After extracting the textual patterns, we categorize them  into the defined 8 types based on the rules. There will be one to many relationship between jobs and types. For example, if a job have failed tests and skipped tests, it will be tagged with both “failed test” and “skipped or pending test” types.


%\textit{Distinguishing the bugs introduced before CI adoption and the bugs introduced after CI adoption}: 
%We suppose that the defects before and after introducing Travis-CI are different. To verify this, we use the SZZ algorithm to gather bug data, and divide them into two groups ( \ie bugs introduced before Travis-C and bugs introduced after Travis-CI ).
%In GitHub, when a bug is reported, an issue will be created to track this bug, and subsequently a set of commits occur for bug fixing. 
%We decide whether a issue is a \textit{fix issue} based on the constrains that 1) the issue has been taged as a bug (with labels like \textit{bug}, \textit{defect}, \textit{fault}); 2) the issue has at least one commit with source file modification to fix this bug. To do this, We first analyze the commit messages to recognize its corresponding issue based on the special textual patterns (\eg fix \textit{issue\_id}, close \textit{issue\_id}, gh-\textit{issue\_id}), and then check if this issue has bug tag. If yes, we will use git diff to find the change details (\ie changed files and lines), and further check if the commit has changed source files.
%After the above conditions checking, we identify fix issues and the corresponding fix commits. The lines changed in fix commits are targeted to fix the bug. Using \textit{git blame}, we can locate which commit last added or modified these lines. We consider this commit as a buggy commit as it introduced bug. 
%As we know, a fix issue will only be triggered by the bug introduced before the issue is opened. Therefore, a necessary condition for filtering is that the buggy commit must have been pushed before the bug being reported (\ie issue being created), otherwise, this buggy commit should not be implicated.
%With above process, the bugs can be divided into two groups based on when the bugs were introduced (\ie buggy commit time) and when the project started using Travis-C. For each group, we gathered the bug logs from the fix issues and fix commits, and then built word cloud to analyze the differences between them.

\subsection{Data Description}
% !TEX root = CI_Adoption.tex



\begin{table}[]
	\centering
	\caption{Summary statistics for 1566 studied projects}
	\label{projs_summary}
	\begin{tabular}{ p{1.2cm} p{0.7cm} p{0.7cm} p{0.7cm} p{0.7cm} p{0.7cm} p{0.7cm}}
		\hline
		Statistic       & Min            & 1st Qu.        & Median          & Mean           & 3rd Qu.        & Max             \\ \hline
		
		created\_at     & 2008/6/6 & 2011/8/8  & 2012/7/30 & 2012/6/27 & 2013/6/6  & 2014/11/7 \\
		size            & 500            & 1594           & 6006            & 40460          & 26980          & 2433000         \\
		
		\# stars        & 0              & 6              & 33.5            & 344.1          & 180.8          & 18480           \\
		\# forks        & 0              & 4              & 18              & 133.2          & 74             & 7247            \\
		
		\# commits      & 5              & 184.2          & 412             & 1385           & 1085           & 118000          \\
		\# issues       & 0              & 2              & 20              & 125.9          & 89.75          & 10330           \\
		
		\# PRs & 0              & 2              & 14              & 97.68          & 64             & 7715            \\
		\# builds       & 1              & 39             & 99.5            & 347            & 278            & 16700          
	\end{tabular}
\end{table}


Our study makes use of a large-scale data set collected from the selected 1566 java projects. Table~\ref{projs_summary} summarizes the descriptive information, including the min, median, mean and max values. All these projects have adopted Travis-CI, with an average number of 347 builds. We further census the builds in terms of the event type and final state. Table ~\ref{builds_count} shows that, there are total 541959 builds in our data set, 68\% come from push commit and 32\% come from pull request. Of these builds, most were passed (67.1\%), 18.6\% were failed, and the others were errored or canceled. Besides, Table ~\ref{count_info} lists the statistics for \# commits, \# issues and \# PRs before and after adoptiong CI respectively. We can preliminarily see that, in our studied projects, there are less commits after CI adoption, but more issues and PRs. From the summary statistics, we found there are large variances between projects in terms of each attribute, as shown in Table ~\ref{projs_summary}. For example, 18\% projects have a value of \# issues eaqual to 0, as they haven't featured issue tracking yet. When studying the evolution practices on issues, these projects without issues will bias our conclusion. As a result, to avoid too many zeros in the following time series analysis, we did more data filtering on the 1566 projects for each RQ individually based on different conditions as follows.

For RQ1 and RQ2,we study the code churn and commit frequency. In our data set, 8.9\% commits are merge commits. First we remove these merge commits, as they were automatically generated by Git by default. Then, we further filter the projects based on the number of nonmerge commits. Each project should have at least 500 nonmerge commits. Finally, the data set of RQ1 and RQ2 consists of 567 projects with total 1629090 nonmerge commits.

For RQ3, we selected the projects with at least 100 issues. And then we got 293 projects with total 143573 issues.

For RQ4,  we investigate the changes in \# tests per build.
As introduced above, we collected \# tests from build logs. So we first filtered the projects based on the number of builds. Each projects should have at least 100 builds. This selection left us 736 projects. 
During the collection of \# tests, we found that 219 of them did not deploy test excecution. We just removed these projects as they didn't have \# test values. Among the other projects, 250 of them have more than 90\% builds with at least one test. Here, we only consider these 250 projects with high coverage of testing ( \textgreater{ 90\%} ).

\subsection{Time Series Analysis Methods}

We use data visualization and statistical modeling methods to discover longitudinal patterns indicative of CI adoption effects on development practices.
As one of the contributions of this paper, we introduce the statistical modeling framework of {\emph regression discontinuity design} to assess the existence and extent of a longitudinal effect of CI adoption on development practices.

To evaluate the effect of a treatment, e.g., a new drug on a disease progression, randomized experimental trials are typically used, in which the experimental cohort is randomly split into a treatment group, i.e. those given the drug, and a control group, i.e., those given an experientially identical placebo. Then, the effect is evaluated based on the difference in disease progression between the two groups.
In the absence of randomized trials, as is the case with trace data common in empirical software engineering, weaker techniques, e.g., quasi-experiments, are employed which make additional assumptions.

Regression discontinuity design (RDD) is an example of such a technique, used for modeling the extent of a discontinuity of a function between its values at points just before and just after a given intervention point. 
It is based on the assumption that in the absence of the intervention, the trend of the function would be continuous.
A common situation where a discontinuity occurs is illustrated in Fig.~\ref{RDDIllustration}, where the discontinuity is shown as occuring at the time of the intervention, and is manifested as a different regression before and after that point.

There are a number of different formalizations of RDD, most prominently sharp RDD and fuzzy RDD.~\cite{}
Each can, in turn, be implemented in a variety of ways.
To model the effect of CI adoption on developer practices, here we chose one of the simpler approaches of linear regression with an interaction term.
We summarize next our approach, following the description from a recent review of RDD~\cite{}. We refer to Figure~\ref{RDDIllustration} for an illustration.
Let $Y$ be the outcome variable in which we are looking for a discontinuity, e.g. commit churn per month, let $X$ be the temporal variable containing the time of intervention, and $c$ be the time point at which the intervention, in this case CI adoption has happened.
Then, an RDD model for points $x_i$ in equal intervals $h$ on each side of $c$, $c-h \le x_i \le c+h$, is given by:

\[y_i \ = \alpha + \beta(x_i-c) + \gamma w_i + \delta(x_i-c)w_i + \epsilon_i,\]

\noindent where $w_i = (x_i > c)$, i.e. $w_i$ is 1 if point $x_i$ is included in the treatment group (e.g., after CI adoption), and 0 if it is before the treatment.
In fact, this model encapsulates two separate regressions.
For points before the tratment, the resulting regression line has a slope of $\beta$, and after the treatment $\beta + \delta$.
The size of the effect of the treatment is the difference between the two regression values of $y_i$ evaluated at $x=c$, and can be seen to be equal to the intercept $\gamma$.

For example, in Fig.~\ref{RDDIllustration}, ...

For each project, we use a sharp RDD model implemented as the above double linear regression, on data centered at the time of CI adoption, and having equal number of points on each side.
Solving the regression gives us the coefficients, which we can reason about, especially the effect size, if any.
 




