% !TEX root = CI_Adoption.tex
\section{Methods}
\label{sec:method}

a) data gathering

From GITHUB Search API, we find over 430k Java projects created before 2015-05-01 and with size larger than 500 kilobytes. After consulting Travis-CI API, we further identify projects that: 1) adopted Travis-CI; 1) at least 9 months before and 9 months after Travis-CI adoption. After this filtering process, our data set contains 1581 projects. 

For these projects, we collect the whole history of commits, issues, pull requests and comments from GITHUB API, and the history of Travis-CI builds and job logs from Travis CI API.  The data were extracted from the inception until 2016-05-01.

b) data description

c) Time Series Analysis Methods

We use data visualization and statistical modeling methods to discover longitudinal patterns indicative of CI adoption effects on development practices.
As one of the contributions of this paper, we introduce the statistical modeling framework of {\emph regression discontinuity design} to assess the existence and extent of a longitudinal effect of CI adoption on development practices.

To evaluate the effect of a treatment, e.g., a new drug on a disease progression, randomized experimental trials are typically used, in which the experimental cohort is randomly split into a treatment group, i.e. those given the drug, and a control group, i.e., those given an experientially identical placebo. Then, the effect is evaluated based on the difference in disease progression between the two groups.
In the absence of randomized trials, as is the case with trace data common in empirical software engineering, weaker techniques, e.g., quasi-experiments, are employed which make additional assumptions.

Regression discontinuity design (RDD) is an example of such a technique, used for modeling the extent of a discontinuity of a function between its values at points just before and just after a given intervention point. 
It is based on the assumption that in the absence of the intervention, the trend of the function would be continuous.
A common situation where a discontinuity occurs is illustrated in Fig.~\ref{RDDIllustration}, where the discontinuity is shown as occuring at the time of the intervention, and is manifested as a different regression before and after that point.

There are a number of different formalizations of RDD, most prominently sharp RDD and fuzzy RDD.~\cite{}
Each can, in turn, be implemented in a variety of ways.
To model the effect of CI adoption on developer practices, here we chose one of the simpler approaches of linear regression with an interaction term.




 




