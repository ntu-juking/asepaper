% !TEX root = CI_Adoption.tex
\section{Examples}
\label{sec:examples}
Before diving into quantitative analysis we eyeball several projects in our collection. 
We observe that on several occasions continuous integration has been adopted as
part of a larger scale restructuring effort. 
For instance, \texttt{allendevco/pill-logger} has started using Travis-CI on November 20,
2013; in the same period developers have changed the project build system from ant to
gradle and as a consequence of that changed the way the project folders are organized.
Similarly, \texttt{Netflix/denominator} has introduced continuous integration on March 13, 2015
after an active development period of daily commits starting on March 3.
On the very same date continuous integration has been introduced 
the project updated library dependencies and restructured tests, while
on the next day it added examples and removed support of DiscoveryDNS, and continued
this rather active development till the version 4.5.0 has been released on April 6, 2015.
The project has three additional commits in April and then went to a recess till early August. 
Examples such as \texttt{allendevco/pill-logger} and \texttt{Netflix/denominator} 
suggest that the software development immediately prior to introduction of continuous 
integration and immediately after it might not be representative for the project.
Therefore, in the quantitative analysis we perform in Section~\ref{XXX} we have decided to
exclude one month before the first Travis-CI build and one month after it.