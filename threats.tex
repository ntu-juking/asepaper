% !TEX root = CI_Adoption.tex

\section{Threats to Validity}
\label{sec:threats}
We focus construct and internal validity~\cite{perry2000empirical}, as we 
do not claim generalizability.

\smallskip \emph{Construct Validity:} %assesses whether the variables we 
%considered accurately model the constructs of our study. 
One of our constructs is a project's ``CI adoption time,'' which 
we operationalize as ``first commit to \texttt{.travis.yml}''. %``first \Tvis build''. 
A more precise definition of would have involved 
reconciling two additional timestamps: registration of the repository with \Tvis, and 
first build
%, and the addition of the \texttt{.travis.yml} configuration file to the
%repository root folder.
This is due, \eg, to the fact that \Tvis can also start a build using some default 
environment settings (Ruby) \emph{without} \texttt{.travis.yml} being present.
%However, by default Travis-CI is configured for Ruby, and obviously any attempt to build a Java project as if it were Ruby are deemed to fail.
%When registering the project one might also indicate that the builds should be created only if \texttt{.travis.yml} is present.
%Similarly, even after \texttt{.travis.yml} is present in the repository, \Tvis may 
%not necessarily immediately start a build; for that to happen the repository 
%has to be registered with \Tvis and some changes have to be pushed to a
%surveilled branch, either directly or as part of an open pull request. 
%Information about a repository's registration with \Tvis is not publicly available. 
%However, in practice most projects perform all three actions on the same day, 
%or at least within a couple of days.
%As an illustration, we point the reader to the singular exception we encountered
%in our informal explorations of the data, \texttt{SuperAwesomeLTD/sa-mobile-sdk-android},
%for which completing all three actions took more than 2 years.
%However, in several projects such as \as{Bogdan, what was the name of this SuperAwesome thing?} completing the three actions has taken more than \as{Bogdan, please check} years.
Hence, validity of the ``CI adoption time'' construct might have been threatened 
by our operationalization.
We reduce this threat by excluding the period immediately before and after
our $t=0$ from all analyses.

Another construct %whose validity might have been threatened 
is ``size of a code change''.
We operationalize this as the number of churned lines, customarily 
interpreted as the sum of the number of added and removed lines~\cite{GigerPG}. 
In this way, moved lines are counted twice, as being added and as being removed.
Moreover, we do not distinguish between lines of source code and other lines, 
%The reason is that while the repositories considered are labeled by \GH as 
%Java repositories and \emph{predominantly} contain Java code, they also 
%contain source code in other languages, configuration files, and documentation. 
since it is non-trivial when dealing with many languages.
%Distinguishing between source code and non-source code in all %those 
%languages in our sample is non-trivial.

%The validity of our measurements, such as the number of commits or the 
%number of builds, might have been threatened by the limitations of the \GH 
%and \Tvis APIs.
%Furthermore, validity of the error type recognition might have been affected 
%by the selection of the keywords in the error type recognition tool. 
%To mitigate this threat and encourage replications, we will make our tools 
%and data publicly available. (\emph{Note: to avoid complications related to the double-blind MSR policy, and on advice of the MSR PC chairs, we delay the release until publication}) %\footnote{\url{LINK}}
%\footnote{\url{http://web.cs.ucdavis.edu/~filkov/software/adoptCI/}}

%Finally, in this work we use terms ``project'' and ``repository'' interchangeably, which is not always the case~\cite{Kalliamvakou2014Promises}. 

%is the extent to which the data was collected correctly. We collected the data of commits, issues and PRs from GITHUB. After the data being collected, we recheck to make sure we didn't miss any commits, issues and PRs. For example, we searched the issues using GitHub search API for each project, and compared the returned \textit{total\_count} field with the total number of issues we have collected. If they were different, we found the missing issues, and regathered them until all the issues were gathered. Similarly, the data collected from Travis-CI was also rechecked carefully. To ensure the reliability of test number per build, we randomly selected a sample of builds, manually reviewed the log files and verified the data from manual review and automatical collection to make sure they are consistent. Similar, the data of build error classification was also examined in this manually checking way.

\smallskip \emph{Internal Validity}: %is the extent to which the conclusions can 
%be drawn from our measurements. 
To reduce these threats we have opted for RDD~\cite{imbens2008regression}, 
a sound approach to statistical modeling of discontinuity in time series. 
Application of RDD to SE data has been recently advocated 
by Wieringa~\cite{Wieringa}.
%We varied 
%\as{Do we have expect to have confounds somewhere? Not in RDD but maybe in the \# tests study?}
% about the causal effect of independent variables on the dependent variables. We considered if commit churn, number of tests, build error types and number of issues will evolve in a certain trend after CI is adopted. For each perspective, we filtered the projects based on particular required conditions, and focus on the individual variable, without considering whether other variables will influence the trend. If such variables exist, they may weaken the credibility of our results.
 
%The similar patterns in all our tables describing the RDD results, of higher upward trends before and higher downward trends after CI adoption, may bring on the question whether the methods are biased. To address that concern, we ran our methods on randomized matrices of our data, 200 times for each model. The results on such randomized data showed equal proportions of up to down trends before and after CI adoption, thus eliminating this concern.
 
 
The attentive reader will have also noted multiple data filtering 
steps described in the previous sections.
We tested the robustness of our models by varying the different data filtering 
criteria (\eg 9 month windows instead of 12, projects kept in the sample even 
if not active all 24 months), and we observed similar phenomena.
%While we cannot claim 
%
% did not explicitly test how our results would differ under different
%assumptions and data filtering criteria, if at all, we did base our decisions on
%statistical best practices (\eg regarding data sparsity) as well as 
%earlier studies of \Tvis~\cite{era14,% 
%VasilescuYWDF15, yue2015wait, BellerGZ16, Hilton2016, Yu2016}.
%%our prior experience with mining \GH and \Tvis data~\cite{yue2015wait, VasilescuYWDF15}.
We encourage independent replications to further assess the robustness of our results.
 
%In addition, we selected the projects indeed with CI adoption, but didn't take into account the CI usage frequency. If a project has only a few builds within a few years, the software engineering practices may be different from other projects which use CI frequently. 
%Besides, in our data set, a group of builds didn't have test execution descriptions in the log files. Most of them are due to the projects without featuring test execution (nearly 30\% projects). For the other builds, the possible reasons are that (1) the log files are blank; (1) the build was canceled before running tests; (3) all the jobs were errored or failed and then exited before running tests; (4)the build was set to skip tests (\eg \textit{DskipTests=true}). To solve this, we selected the projects which featured test execution and had over 90\% builds with at least one test.


%\smallskip \emph{External Validity} pertains to generalizability of our findings 
%beyond the collection of \GH projects we investigated.
%While we do expect our findings to apply to other mid to large-sized projects,
%this expectation might be threatened by peculiarities of other CI tools (\eg Jenkins, 
%Hudson) as well as differences between open-source and proprietary projects.

%the projects in our study are all Java open source projects from GitHub repository. Therefore, our conclusion may not be generalized either to projects developed by other programming languages or projects that are not open-source. Besides, as we know, there have been multiple CI techniques in practice, \eg Travis CI, Jenkins CI, Hudson CI. In our study, we only focused on Travis CI, which may also threaten the generalization of our results. However. as Travis-CI is the most popular CI service in GitHub, we believe our results on Travis-CI are representative for CI.  This is an inherent threat for all studies in empirical software engineering, as many factors cannot be characterized completely. To mitigate this threat, there is a need to replicate our study using a wide variety of systems and considering other CI techniques in the future work.
