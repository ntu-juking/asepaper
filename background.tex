% !TEX root = CI_Adoption.tex

\section{Development of Research Questions}
\label{sec:background}

For a developer not proficient with the operational side of the process, 
transitioning to an integrated CI platform, like \Tvis, involves adaptation 
of their established processes to the new environment. 
During this transition, some developers will experience a more streamlined 
process evolution trajectory than others. 
Studying those trajectories can provide lessons learned.


%We expect the following to potentially change in the transition <need to write theory behind each>:
%
%On the developer side:
%Change in code writing/committing: we expect smaller change sets over time
%We expect More unit testing over time
%Operations side:
%More discussions in code review over time
%Different categories of initial faults

Continuous integration encourages developers to ``break down their work 
into small chunks of a few hours each'', as smaller and more frequent commits 
helps them keep track of their progress and reduces the debug effort~\cite{Fowler,Duvall}. 
%Duvall [p. 31,38,40]
Based on these guidelines we formulate our \textbf{RQ1}: 
Are developers reducing the size of code changes in each commit post CI adoption? 
Do they continue to do so over time?

The size of the commit is closely related to commit frequency: indeed, the 
aforementioned quote from Fowler's blog refers to ``chunks of a few hours each''. 
In an early study Miller has observed that, on average, Microsoft developers 
committed once a day, while off-shore developers committed less frequently 
due to network latencies~\cite{Miller}; Stolberg expects everyone to commit every 
day~\cite{Stolberg} and Y\"{u}ksel reports 33 commits per day after introduction 
of CI~\cite{Yuksel}. 
Hence, we ask in \textbf{RQ2}: Are developers committing code more frequently?

For CI, and \DO in general, to have the stated benefits, effective communication on arising issues is important.
Although not stated in the proposed "best practices" by Fowler, the frequency of opened issues per time period may change after CI adoption to meet the demands arising from changes in other practices, like code churn and automated testing.
%In the very popular pull-request model of development, code review is done through open issues.
%\as{Can we say something about the popularity of the pull-request model of development in GH? Is this your work or GG's?} 
%However, maintaining the process of code review has been recognized as one of 
%the challenges when adopting continuous deployment~\cite{ClapsBSA}, software 
%development technique related to continuous integration.
Hence, we ask in \textbf{RQ3}: 
Are developers transitioning to opening more or fewer issues after the adoption of CI?

Finally, CI is closely related to presence of automated tests~\cite{Fowler}. 
Duvall even claims that CI without such tests should not be considered CI 
at all~\cite{Duvall}, while Cannizzo, Clutton, and Ramesh deem an extensive 
suite of unit and acceptance tests to be an essential first step~\cite{CannizzoCluttonRamesh}. 
Moreover, CI is frequently introduced together with test automation~\cite{Yuksel}.
However, developing tests suited for automation requires change in developers' 
mind set and presence of a comprehensive set of tests incurs additional maintenance 
costs~\cite{CoramBohner}.
Naturally, more automated testing will expose more errors, and CI makes it possible 
to track and react to different error types differently~\cite{BellerGZ16}.
%\as{Y\"{u}ksel reports increase of the number of automated tests but they have combined introduction of CI with automated testing~\cite{Yuksel}. }
Therefore, we formulate \textbf{RQ4}: 
How does automated testing usage change over time after CI adoption? 
And how do build errors evolve over time? 
%\as{Not sure about the formulation; the word ``transition'' implies that in the beginning the developers do not use automated testing. Do we really check this?}

%\subsection{RQs}
