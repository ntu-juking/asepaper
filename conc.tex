% !TEX root = CI_Adoption.tex

\section{Conclusion}
\label{sec:conc}

This paper focused on CI: while several guidelines exist, relatively little 
has been done to evaluate the state of practice.
We empirically studied the impact of adopting \Tvis on development practices 
in a collection of \GH projects, and surveyed the developers responsible for 
introducing \Tvi. %in the projects we have studied.

We observe that the reality of adopting \Tvis is much more complex than suggested by previous work.
The increasing number of merge commits aligns with the the ``commit often'' guideline, but is likely to be further encouraged by the shift to a more distributed workflow with multiple branches and pull requests.
The ``commit small'' guideline, however, is followed only to some extent and the differences between the projects are more important than adherence to this guideline.
As expected, we have observed the increasing trend in the number of issues closed; however, it was surprising that this trend slows down after \Tvis is introduced.
In terms of testing, we observe that after the expected adjustments required by the automated testing, the debugging complexity seems to increase. 
The most interesting observations were related to the pull request resolution: the expected increasing trend in the number of pull requests being closed manifests itself after the introduction of \Tvis, and even then only after the initial plateau period. 
This observation calls for a more profound investigation of how \GH teams change their behavior in response to the introduction of \Tvis. 
Finally, we have observed the pull request latency increasing  despite the code changes getting smaller.
%\as{till here}
%We observed that for most projects no significant linear trends 
%are visible, neither before nor after introduction of \Tvis.
%In cases when such a linear trend can be observed it is more often decreasing 
%(fewer churned lines, less frequent commits, fewer issues). 
%And while decrease in the churned lines concurs with one of the aforementioned 
%``best practices'', decrease in the commit frequency or number of issues seems 
%to disagree with them. 
%Finally, we considered trends in testing and the types of errors revealed 
%by the automatic testing.
%The number of tests run by \Tvis shows an increasing trend, 
%%\as{VF, please check this once you have done a more careful modeling} 
%in agreement with the importance of automated testing in CI.
%We also observed increase in the percentage of compilation and execution errors 
%as well as failed and skipping tests as opposed to missing files or dependencies.
%We conjecture that as file omissions are easy to fix, once such an error has been 
%observed it is immediately fixed, in contrast to repeated attempts to fix other 
%kinds of errors causing repeated build failures.
%More work has to be done to tease out the (possibly subtle) interactions in the 
%interleaved processes of building, coding, and testing.
%%As developers are getting more and more used to the rapid feedback provided by the CI 
%%\as{This still does not explain the evolution; the conjecture that the developers are running more and more attempts to fix a bug seems to be contradicted by 
%%our conclusion on \# commits (unless they are using pull requests...)}

While CI is a popular mechanism, we do not 
find overwhelming evidence that developers transition to ``best practices" 
following its adoption.
In fact, it is apparent that the reality of the \Tvis adoption is more complex than suggested by the previous studies.
Project specific concerns may guide individual implementations, usage, 
and practices.
Moreover, the ways developers benefit from the introduction of \Tvis do not necessarily correspond to their
expectations, \eg unstable performance of \Tvis allowed one of the survey respondents to detect hard-to-find bugs.
%These findings have support in recent qualitative studies, where CI needs and  wishes of open-source and proprietary code developers have been surveyed~\cite{Hilton2016, hilton2016continuous}.
We consider investigating longer term impact of \Tvis as future work.
%%\as{Danny Dig and co-authors have only compared Travis with no-Travis but did not look at the transition.}