% !TEX root = CI_Adoption.tex

\section{Conclusion}
\label{sec:conc}

This paper focused on continuous integration: while several ``best 
practices'' have been proposed, relatively little has been done to evaluate the 
state of practice.
Using a collection of \GH Java projects we conducted an 
empirical study of the impact of adopting \Tvis on development practices.
We observed that for most projects no significant linear trends 
are visible, neither before nor after introduction of \Tvis.
In cases when such a linear trend can be observed it is more often decreasing 
(fewer churned lines, less frequent commits, fewer issues). 
And while decrease in the churned lines concurs with one of the aforementioned 
``best practices'', decrease in the commit frequency or number of issues seems 
to disagree with them. 
Finally, we considered trends in testing and the types of errors revealed 
by the automatic testing.
The number of tests run by \Tvis shows an increasing trend, 
%\as{VF, please check this once you have done a more careful modeling} 
in agreement with the importance of automated testing in CI.
We also observed increase in the percentage of compilation and execution errors 
as well as failed and skipping tests as opposed to missing files or dependencies.
We conjecture that as file omissions are easy to fix, once such an error has been 
observed it is immediately fixed, in contrast to repeated attempts to fix other 
kinds of errors causing repeated build failures.
More work has to be done to tease out the (possibly subtle) interactions in the 
interleaved processes of building, coding, and testing.
%As developers are getting more and more used to the rapid feedback provided by the CI 
%\as{This still does not explain the evolution; the conjecture that the developers are running more and more attempts to fix a bug seems to be contradicted by 
%our conclusion on \# commits (unless they are using pull requests...)}

While CI is a popular mechanism, we don't 
find overwhelming evidence that developers transition to ``best practices" 
following its adoption.
In fact, it is apparent that the ``one size fits all" approach may not apply here, 
and that project specific concerns may guide individual implementations, usage, 
and practices.
These findings have support in recent qualitative studies, where CI needs and 
wishes of open-source and proprietary code developers have been surveyed~\cite{Hilton2016, hilton2016continuous}.
We consider investigating whether introduction of CI in a project leads to 
realization of those expectations, independent of Fowler's ``best practices," 
as future work.
%\as{Danny Dig and co-authors have only compared Travis with no-Travis but did not look at the transition.}