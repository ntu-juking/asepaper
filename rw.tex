% !TEX root = CI_Adoption.tex

\section{Related Work}
\label{sec:rw}

Adoption and impact of CI have attracted quite some attention from the research 
community already. 
The earlier studies interpret CI as \emph{distributed development and obligation 
to integrate one's own contributions}, as common in most open-source software.
Though not directly comparable, we review these studies for historical reasons.
Holck and J{\o}rgensen, based on studies of FreeBSD and Mozilla, observed that 
CI (under this definition) is capable of replacing traditional software engineering 
coordination mechanisms~\cite{HolckJ03}.  
Deshpande and Riehle studied adoption of CI based on Ohloh.net data~\cite{Deshpande2008}
under the assumption that it reduces the size of commits (to facilitate coordination
in distributed settings), but observed no commit size reduction in their 5,122 projects. 
%Assuming that the use of CI (again under this early definition) reduces the 
%size of commits, the authors studied the size of commits in 5,122 projects and, 
%since no commit size reduction was observed, they concluded that CI has not 
%been used. 
%Their observations might have been affected by lack of actual relation between 
%presence of continuous integration and commit size, aggregation effect of considering the average commit size over different projects as well as the period studied (January 1990--December 2006). 

More recently, the ease of use of the \Tvis~\cite{TravisCI} system led to its 
popularity on \GH, and triggered a series of research studies~\cite{era14, 
VasilescuYWDF15, yue2015wait, BellerGZ16, Hilton2016, Yu2016}.
This line of research is closer to the current work as it performs empirical
analysis of \Tvis data. Moreover, similarly to Vasilescu \etal~\cite{era14}
and Hilton \etal~\cite{Hilton2016} our work can be seen as related to adoption of \Tvis,
and similarly to Beller \etal~\cite{BellerGZ16} we study build failures.
However, all these papers compare projects that adopt CI with those that do not, 
or different kinds of projects that adopt CI with each other, and therefore, their
conclusions might be threatened by specific aspects of individual projects.
In our work we therefore opt for the evolution perspective and focus on the
differences between the development practices within the project before and
after adoption of \Tvis.
Hence, while the previous research has focused on the \emph{between-subject 
design} we provide complementary insights through a \emph{within-subject design}.
%\as{Bogdan, Vladimir, please check the previous paragraph.}

Gousious \etal\ have studied work practices and challenges in pull-based
development on \GH~\cite{gousios2015work, gousios2016work}. 
They report that 75\% of the projects surveyed use CI tools to evaluate
code quality~\cite{gousios2015work} and that more than 60\% of the surveyed 
contributors employ automatic testing, either locally or using a CI server~\cite{gousios2016work}. 
Importance of tooling facilitating the testing tasks has been already recognized by 
Pham \etal~\cite{pham2013creating}: back in 2012 when the authors have conducted
the interviews, \Tvis has only started to support mainstream languages such as Java 
and lack of such tooling has been reported as an important challenge.
These findings further emphasize importance of CI in modern software development. 

%Already a 2009 Forrester report indicated that 67\% of the 52 development professionals
%who adopted Agile adopted continuous integration, another 19\% were in process 
%of implementing it and another 10\% planned to implement it~\cite{Forrester}. 
March 2016 survey of 1,060 IT professionals indicated that 81\% of the enterprises 
and 70\% small and midsize businesses implement \DO~\cite{rightscale}. 
Not surprisingly, the topic of industrial adoption of continuous integration 
has also attracted substantial research attention~\cite{hilton2016continuous,Leppanen2015,Laukkanen2015Agile,Debbiche2014,Stahl2014ICSEComp,Stahl2014JSS,Stahl2013Experienced} and is a subject of the recent literature survey by Eck, Uebernickel and Brenner~\cite{EckUB14}. 
However, this line of work is based on interviews or surveys rather than on analysis of the development data
and as such is to larger extent susceptible to the perception bias: \eg Leppanen \etal\ report that
CI introduction is beneficial for productivity~\cite{Leppanen2015}, 
Eck \etal\ stress that productivity is likely to decrease before any positive effects can be observed~\cite{EckUB14},
and St{\aa}hl and Bosch validated this hypothesis only partially~\cite{Stahl2013Experienced}.

Several papers have reported on experiences with applying CI in specific domains:
\eg Zaytsev and Morrison report that introduction of CI in neuroinformatics
software development facilitated integration of new changes, and hence reduced 
time-to-marked without compromising on quality~\cite{Zaytsev:Morrison}; 
and M{\aa}rtensson \etal\ observed that long build- and test-time in software-intensive
embedded systems discourage developers from frequent committing~\cite{Martensson2016}.

%\as{Miller~\cite{Miller} discusses build breaks, is this relevant?}
%\as{Van der Storm discusses backtracking as a way of addressing build failures~\cite{Storm2008}.}
%\as{Laukannen and M\"{a}ntyl\"{a} have surveyed three papers discussing the impact of the build waiting times. Some of the consequences of the long/short build waiting times are related to CI~\cite{Laukkanen2015RCSE}.}

Automated testing is an important factor that affects the cost-effectiveness of continuous integration. In recent years, much effort has been devoted to improve the quality and efficiency of automated testing in continuous integration. Campos et al.~\cite{campos2014continuous} used Continuous Test Generation (CTG) to enhance continuous integration with automated test generation. Their experimental results showed that CTG could lead to a significantly higher code coverage while reducing the test generation time. Elbaum et al.~\cite{elbaum2014techniques} and Hu et al.~\cite{hu2016Implementation} applied test case selection and test case prioritization to test suites, making continuous integration processes more cost-efficient. D{\"o}singer et al. ~\cite{dosinger2012communicating} proposed the continuous change impact analysis technique to analyze the dependencies between the codes of various projects to improve the effectiveness of automated testing. Long et al.~\cite{long2015collaborative} designed tools to support collaborative testing between testers. Nilsson et al.~\cite{nilsson2014visualizing} developed a technique to visualize end-to-end testing activities in the continuous integration processes. All these efforts in automated testing improve the cost-effectiveness of continuous integration. 

 