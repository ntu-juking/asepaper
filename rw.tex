% !TEX root = CI_Adoption.tex
\section{Related Work}
\label{sec:rw}
Adoption and impact of continuous integration have attracted \as{limited/extensive} attention from the research community. Already in 2003 based on the studies of FreeBSD and Mozilla Holck and J{\o}rgensen have observed that continuous integration, interpreted as distributed development and obligation to integrate one's own contributions, is capable of replacing traditional software engineering coordination mechanisms~\cite{HolckJ03}.  

\as{Miller~\cite{Miller} discusses build breaks, is this relevant?}
\as{Van der Storm discusses backtracking as a way of addressing build failures~\cite{Storm2008}.}

Deshpande and Riehle have studied adoption of continuous integration based on Ohloh.net data~\cite{Deshpande2008}. Assuming that the use of continuous integration reduces the size of commits, they have studied the size of commits in 5122 projects and, since no commit size reduction has been observed, they have concluded that continuous integration has not been used. Their observations might have been affected by lack of actual relation between presence of continuous integration and commit size, aggregation effect of considering the average commit size over different projects as well as the period studied (January 1990--December 2006). A more careful investigation 