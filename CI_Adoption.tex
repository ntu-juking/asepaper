\documentclass[conference]{IEEEtran}
\usepackage{cite}
\usepackage{balance}


\usepackage{xspace}
\newcommand{\ie}{{\emph{i.e.}},\xspace}
\newcommand{\viz}{{\emph{viz.}},\xspace}
\newcommand{\eg}{{\emph{e.g.}},\xspace}
\newcommand{\etc}{etc.}
\newcommand{\etal}{{\emph{et al.}}}

\usepackage{amssymb}
\newcommand{\nnbb}[2]{
    \fbox{\bfseries\sffamily\scriptsize#1}
    {\sf\small$\blacktriangleright$\textit{#2}$\blacktriangleleft$}
   }

\newcommand{\as}[1]{\nnbb{Alexander}{#1}}
\newcommand{\bv}[1]{\nnbb{Bogdan}{#1}}
\newcommand{\vf}[1]{\nnbb{Vladimir}{#1}}
\newcommand{\yz}[1]{\nnbb{Yangyang}{#1}}

\begin{document}

\title{Evolution of Software Development Practices
Following Adoption of Travis CI: A Large-scale Empirical Study of GitHub Traces}

\author
{\IEEEauthorblockN{}
\IEEEauthorblockA{}
}
\maketitle
\begin{abstract}
Continuous Integration (CI) is a key part of the devops ideology of mashing development and operations together to shorten the cycles of delivering a product to users. CI has become a disruptive innovation in software development: with proper implementation, e.g. Travis CI or Jenkins CI, positive effects have been demonstrated on pull request throughput and scaling up of project sizes. As any other innovation, adopting it implies adapting existing practices in order to take full advantage of its potential. Here we study the adaptation and evolution of code writing, review practices and unit testing practices as Travis CI is adopted by XX established projects on GitHub. By employing a mix of quantitative studies and case studies we triangulate the general evolution trajectories, and provide reasoning for the differences encountered among the projects.
\end{abstract}

\section{Introduction}
Devops, or bringing development and build/release activities in the same framework can bring changes in the product to the user-plane more quickly. For it to be effective, the technology that implements these ideas has to allow for a seamless back and forth between development, integration, code testing review, and release. 

Continuous Integration is the part of devops that seamlessly builds, tests, and integrates developer changes, and performs any pre-specified testing. CI (\eg Travis CI, Jenkins CI, Hudson CI), if implemented properly, can benefit the distributed software development process, in particular code change throughput~\cite{Stolberg} and some aspects of code quality~\cite{VasilescuYWDF15}. It is a disruptive technology, in that it can scale up distributed development without noticeable diminishment in quality.

\section{Theory}

For a developer not proficient with the operations side of the process, transitioning to an integrated CI platform, like Travis CI, involves adaptation of their established processes to the new environment. During this transition, some developers will experience a more streamlined process evolution trajectory than others. Studying those trajectories can provide lessons learned.


We expect the following to potentially change in the transition <need to write theory behind each>:

On the developer side:
Change in code writing/committing: we expect smaller change sets over time
We expect More unit testing over time
Operations side:
More discussions in code review over time
Different categories of initial faults

Continuous integration encourages developers to ``break down their work into small chunks of a few hours each'', as smaller and more frequent commits keep them to keep track of their progress and reduces the debug effort~\cite{Fowler,Duvall}. %Duvall [p. 31,38,40]
Therefore, in \textbf{RQ1} we investigate whether introduction of the continuous integration has indeed led to smaller commits.
\as{Why do not we look at their frequency?}

Moreover, continuous integration is closely related to presence of automated tests~\cite{Fowler}. Duvall even claims that continuous integration without such tests should not be considered continuous integration at all~\cite{Duvall}.

\subsection{RQs}

\section{Methods}

\section{Results and Discussion}


% !TEX root = CI_Adoption.tex
\section{Related Work}
\label{sec:rw}
Adoption and impact of continuous integration have attracted \as{limited/extensive} attention from the research community. Already in 2003 based on the studies of FreeBSD and Mozilla Holck and J{\o}rgensen have observed that continuous integration, interpreted as distributed development and obligation to integrate one's own contributions, is capable of replacing traditional software engineering coordination mechanisms~\cite{HolckJ03}.  
Deshpande and Riehle have studied adoption of continuous integration based on Ohloh.net data~\cite{Deshpande2008}. Assuming that the use of continuous integration reduces the size of commits, they have studied the size of commits in 5122 projects and, since no commit size reduction has been observed, they have concluded that continuous integration has not been used. Their observations might have been affected by lack of actual relation between presence of continuous integration and commit size, aggregation effect of considering the average commit size over different projects as well as the period studied (January 1990--December 2006). 

Ease of use of the Travis-CI~\cite{TravisCI} continuous integration system led to its popularity on GitHub and triggered a series of research studies~\cite{era14,VasilescuYWDF15,yue2015wait,BellerGZ16,Hilton2016,Yu2016}.\as{To be continued. Maybe Bogdan can write about his own work.}

Industrial adoption of continuous integration systems has been recently studied in several papers~\cite{Leppanen2015,Laukkanen2015Agile,Debbiche2014,Stahl2014ICSEComp,Stahl2014JSS} and is a subject of the recent survey by Eck, Uebernickel and Brenner~\cite{EckUB14}. However, this line of work is based on interviews rather than on analysis of the development data.

\as{Miller~\cite{Miller} discusses build breaks, is this relevant?}
\as{Van der Storm discusses backtracking as a way of addressing build failures~\cite{Storm2008}.}
\as{Laukannen and M\"{a}ntyl\"{a} have surveyed three papers discussing the impact of the build waiting times. Some of the consequences of the long/short build waiting times are related to CI~\cite{Laukkanen2015RCSE}.}


\section{Conclusion}

\section{Threats to Validity}

\bibliographystyle{IEEEtran}
\bibliography{references}

\end{document}